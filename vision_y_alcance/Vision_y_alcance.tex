\documentclass{article}

\usepackage[spanish]{babel}

\usepackage[letterpaper,top=2cm,bottom=2cm,left=3cm,right=3cm,marginparwidth=1.75cm]{geometry}

\usepackage{amsmath}
\usepackage{graphicx}
\usepackage[colorlinks=true, allcolors=blue]{hyperref}

\title{Unit Testing - Punto Medio}
\author{Iraís Aguirre Valente}

\begin{document}
\date{}

\maketitle

\begin{abstract}
   Proyecto para la evaluación del segundo parcial en la experiencia educativa Pruebas de Software, impartida por el Dr. Adolfo Centeno Téllez en la Universidad Veracruzana.
\end{abstract}
\date{}

\section{Visión}
Las pruebas unitarias son una técnica para probar software de forma en que las unidades o componentes pequeños de un proyecto de software se prueban individualmente, siendo la unidad una función, un método, un procedimiento, un módulo o un objeto individual. El realizar este tipo de pruebas garantiza que cada unidad del producto está funcionando de manera esperada.

La meta principal de este proyecto de software es implementar una serie de evaluaciones que prueben cada componente de forma individual para comprobar el correcto funcionamiento del proyecto y busca, de igual forma, que la estudiante aprenda a realizar este tipo de pruebas. 

Se espera que al término de este proyecto la estudiante sea capaz de realizar pruebas unitarias en distintas situaciones y siente las bases sobre pruebas de software y su importancia al implementarlas en distintos proyectos.

\section{Alcance}
En el presente proyecto de software se pretende realizar la implementación de pruebas unitarias en ambientes de trabajo reales, utilizanco como medio el framework Angular y el framework Jasmine, cuya integración ya viene configurada en Angular CLI y es un marco de desarrollo basado en el comportamiento para probar código JavaScript.

Para este proyecto se buscará implementar un código que calcúle el punto medio de dos puntos cualesquiera en el plano cartesiano de dos dimensiones, utilizando una fórmula de geometría analítica. Se espera que los componentes implementados en el proyecto de Angular tengan un code-coverage de al menos 80\%, pero se busca que el proyecto alcance un 100\% de líneas de código validadas con éxito bajo un procedimiento de pruebas. 

Para su despliegue se ha utilizado docker y apache, utilizando una máquina virtual de Azure.
\end{document}
